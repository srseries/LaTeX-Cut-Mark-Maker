% Cut & Crop Marks Maker (set by default to A4) - Surendra Rana
% Adapted from an answer on TeX Stack Exchange

\documentclass{article}
\usepackage[paperwidth=210mm,paperheight=297mm,margin=5mm]{geometry}
\usepackage{tikzpagenodes}
\usetikzlibrary {calc}
\usepackage{lipsum}

\begin{document}
\noindent

\title{LaTeX Crop and Cut Mark Sample Text}
\author{from a srseries package}
\date{Last Updated on 28 September 2025}
\maketitle

\noindent
\section{Sample Text}
\begin{tabular}{lr}
  {Lorem ipsum dolor sit amet, consectetur adipiscing elit, sed do eiusmod tempor incididunt ut labore et dolore magna aliqua.} \\ {Ut enim ad minim veniam, quis nostrud exercitation ullamco laboris nisi ut aliquip ex ea commodo consequat.} \\ {Duis aute irure dolor in reprehenderit in voluptate velit esse cillum dolore eu fugiat nulla pariatur.} \\ {Excepteur sint occaecat cupidatat non proident, sunt in culpa qui officia deserunt mollit anim id est laborum.}
  \end{tabular}
  
\begin{tikzpicture}[overlay, remember picture]
    \draw ($(current page text area.north west)+(-0.2,0)$) -- ++ (2.2,0);
    \draw ($(current page text area.north west)+(0,0.2)$) -- ++ (0,-2.2);

    \draw ($(current page text area.north east)+(0.2,0)$) -- ++ (-2.2,0);
    \draw ($(current page text area.north east)+(0,0.2)$) -- ++ (0,-2.2);

    \draw ($(current page text area.south west)+(-0.2,0)$) -- ++ (2.2,0);
    \draw ($(current page text area.south west)+(0,-0.2)$) -- ++ (0,2.2);

    \draw ($(current page text area.south east)+(0.2,0)$) -- ++ (-2.2,0);
    \draw ($(current page text area.south east)+(0,-0.2)$) -- ++ (0,2.2);

\end{tikzpicture}

\end{document}
